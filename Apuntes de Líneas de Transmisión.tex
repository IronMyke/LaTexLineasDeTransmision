\documentclass[12pt]{article}
\usepackage[spanish]{babel}
\usepackage{amsmath}
\usepackage{graphicx}
\title{L\'ineas de transmisi\'on}

\begin{document}

Para el estudio de las LT, tomaremos un elemento diferencial longitudinal en la dirección z.
Asumiremos que la soluci\'on que se propaga por la LT es de tipo TEM, por lo que no existe acoplamiento el\'ectrico ni magn\'etico.
Intentaremos definir el concepto de diferencia de potencial entre 2 puntos situados en la misma secci\'on transversal de la LT
(Suponiendo que en la situaci\'on A, 1 y 2 están en la misma seci\'on, pero en B no)

$$U_{12}|_A - U_{12}|_B = - \oint \vec{D}d\vec{L} = -\int_S \vec{\nabla} \times \vec{E}\vec{n}ds = \int \frac{\partial \vec{B}}{\partial}$$

\section{Modelo Circuital de una LT}
Tanto la tensi\'on como la corriente en un elemento diferencial en una LT dependen \'unicamente de:
\begin{itemize}
	\item De la capacidad creada entre los conductores
	\item Del flujo magnético que atraviesa la secci\'on del elemento (autoinducci\'on)
\end{itemize}

(Ambas definidas por unidad de longitud)
Teniendo esto en cuenta, se puede representar una LT como un cuadripolo con una bobina en serie y un condensador en paralelo.

\section{Ondas de tensi\'on y corriente en una LT}
Aplicando las leyes de Kirchoff:
$$U = L\frac{\partial I}{\partial t}dz + U + \frac{\partial U}{\partial z}dz$$
$$I = L\frac{\partial U}{\partial t}dz + I + \frac{\partial I}{\partial z}dz$$

simplificando, aparece el sistema de ecuaciones conocido como ecuaciones del telegrafista:

$$L\frac{\partial I}{\partial t} = \frac{\partial U}{\partial z}$$
$$C\frac{\partial U}{\partial t} = \frac{\partial I}{\partial z}$$

derivando respecto a z y sustituyendo, llegamos a:

$$\frac{\partial^2 U}{\partial z^2} = LC \frac{\partial^2 U}{\partial t^2}$$

comparando con la ecuaci\'on general de onda, obtenemos la velocidad de propagaci\'on $= \frac{1}{\sqrt{LC}}$

Siguiendo una soluci\'on de D'Alambert, la onda se puede descomponer en sus componentes progresiva y regresiva (que al sumarse producen la onda estacionaria que observamos)
$$U(z,t) = U^+ (z,t) + U^- (z,t) = F_1(t - z/c) + F_1(t + z/c)$$
$$I(z,t) = I^+ (z,t) + I^- (z,t) =\frac{1}{Z_{LT}}( F_1(t - z/c) - F_1(t + z/c) )$$

La velocidad de propagaci\'on, definida  como $\frac{1}{\sqrt{LC}}$ equivale a $\frac{c_0}{\sqrt{\varepsilon_r}}$

La impedancia de la LT se define como el cociente entre $U^+$ e $I^+$, independientemente de tiempo y lugar, y equivale a $\sqrt{\frac{L}{C}}$

Velocidad de propagaci\'on e impedancia caracterizan por completo a una LT.

La potencia transmitida por una LT se calcula como el producto de tensi\'on por intensidad, que desarrollando el producto de sumas de ondas progresivas, nos queda: 
$$P = \frac{(U^+)^2}{Z_{LT}} -\frac{(U^-)^2}{Z_{LT}}$$

Para el estudio de una LT en r\'egimen transitorio (a saber, hallar I y U en cualquier punto de la LT), definimos el r\'egimen transitorio como el breve per\'iodo de tiempo entre dos reg\'imenes permanentes (o estables)

Generaci\'on de una onda incidente (progresiva):
Conectando una LT infinita a un generador, se produce una onda \'unicamente progresiva, cuyo valor de tensión se puede calcular con un simple divisor de tensión (con la resistencia del generador)

Conectamos ahora una LT finita a un generador y a una resistencia en sus extremos:
	en $t=0$, se genera una onda progresiva, que viaja por la LT hasta el final
	en $t=l/v_p$, la onda alcanza la resistencia al final de LT, y para satisfacer la condici\'on de contorno, se genera una onda regresiva ($U^-$), de forma que 
	$$R_L = Z_{LT}\frac{U^+ +U^-}{U^+ +U^-U^+ -U^-}$$
En caso de que la resistencia de carga sea igual a la $Z_{LT}$, el factor de reflexi\'on de la carga es 0, en lo que se conoce como situaci\'on de carga adaptada (que consigue la m\'axima transferencia de potencia) 
$$\rho_L =\frac{R_L - Z_{LT}}{R_L + Z_{LT}}$$
	en $t=2l/v_p$, la onda regresiva alcanza el generador, por lo que se genera otra onda progresiva, según el coeficiente
	$$\rho_g =\frac{R_g - Z_{LT}}{R_g + Z_{LT}}$$

\section{Tema 3: La LT en r\'egimen transitorio}
\subsection{Introducci\'on}
Una LT queda totalmente caracterizada por los valores de $Z_c$ y $V_p$.

En este tema estudiaremos la LT en r\'egimen transitorio, definido como el per\'iodo entre dos reg\'imenes permanentes distintos, provocado por un cambio en las condiciones en un extremo de la l\'inea.
\subsection{Generaci\'on de una onda incidente $U^+$}

Dado un generador de tensi\'on caracterizado por $U_g(t)$ y $R_g$, al que se le conecta una LT, considerada (inicialmente) infinita, de impedancia $Z_c$ y velocidad de propagaci\'on $V_p$, se produce una onda progresiva en $z=0$.
\begin{itemize}
	\item Antes de la conexi\'on al generador, U e I vale ambas 0.
	\item Tras la conexi\'on, la tensi\'on en $z=0$ es la tensi\'on resultante de calcular un divisor de tensi\'on con $R_g$ y $Z_c$: $U^+ = U_g\frac{Z_c}{R_g + Z_c} \hspace{0.5cm} \rightarrow \hspace{0.5cm} U^+ (z, t) = U_g\frac{Z_c}{R_g + Z_c} (t-\frac{z}{V_p})$
\end{itemize}

\subsection{Factores de reflexi\'on y transmisi\'on}
\subsubsection{Factor de reflexi\'on}
Partimos de una LT conectada a un generador en $z=0$ y a una resistencia de carga $R_L$ al final de la l\'inea, en $Z=L$.

En $t=0$, se genera una onda progresiva, como vimos en el apartado anterior, y alcanza la resistencia de carga en $t=\frac{L}{V_p}=T$. Hasta entonces, es la \'unica onda en la l\'inea.

En $t=T$ y $z=L$, observamos las condiciones de contorno impuestas por la carga y por la l\'inea:
$$\frac{U^+(z, t)}{I^+(z, t)} = R_L \hspace{0.5cm} \frac{U^+(z, t)}{I^+(z, t)} = Z_c$$

Puesto que la impedancia de la l\'inea y la resistencia de carga no tienen porqu\'e valer lo mismo, la suposici\'on de que la onda progresiva es la \'unica deja de ser cierta. Se genera pues una onda regresiva.
	$$U(z, t) = U^+ + U^-$$
	$$I(z, t) = \frac{1}{Z_c}(U^+ - U^-)$$
	$$\frac{U(z = L, t)}{I(z = L, t)} = Z_c\frac{U^++U^-}{U^+-U^-} = R_L$$

Definimos pues el factor de reflexi\'on como el cociente entre la onda reflejada y la onda incidente:
$$\rho = \frac{U^-}{U^+} = \frac{R_L-Z_c}{R_L+Z_c}$$

Cuando el factor de reflexi\'on se anule ($R_L=Z_c$), estaremos en una situaci\'on de carga adaptada, y no habr\'a onda reflejada, por lo que toda la onda incidente se transmite.

\end{document}




